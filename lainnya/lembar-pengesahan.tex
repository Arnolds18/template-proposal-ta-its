\chapter*{LEMBAR PENGESAHAN}

% Menyembunyikan nomor halaman
\thispagestyle{empty}

\begin{center}
  % Ubah kalimat berikut dengan judul tugas akhir
  \textbf{KALKULASI ENERGI PADA ROKET LUAR ANGKASA BERBASIS \emph{ANTI-GRAVITASI}}
\end{center}

\begingroup
  % Pemilihan font ukuran small
  \small

  \begin{center}
    % Ubah kalimat berikut dengan pernyataan untuk lembar pengesahan
    \textbf{PROPOSAL TUGAS AKHIR} \\
    Diajukan untuk memenuhi salah satu syarat memperoleh gelar
    Sarjana Teknik pada 
    Program Studi S-1 Teknik Dirgantara \\
    Departemen Teknik Dirgantara \\
    Fakultas Teknik Dirgantara \\
    Institut Teknologi Sepuluh Nopember
  \end{center}

  \begin{center}
    % Ubah kalimat berikut dengan nama dan NRP mahasiswa
    Oleh: \textbf{Elon Reeve Musk} \\
    NRP. 0123 20 4000 0001
  \end{center}

  \begin{center}
    Disetujui oleh Tim Penguji Proposal Tugas Akhir:
  \end{center}

  \begingroup
    % Menghilangkan padding
    \setlength{\tabcolsep}{0pt}

    \noindent
    \begin{tabularx}{\textwidth}{X c}
      % Ubah kalimat-kalimat berikut dengan nama dan NIP dosen pembimbing pertama
      Nikola Tesla, S.T., M.T.          & \\
      NIP: 18560710 194301 1 001        & \\
      &  \\                             & (Pembimbing)\\
      &  \\                             
      % Ubah kalimat-kalimat berikut dengan nama dan NIP dosen pembimbing kedua
      Wernher von Braun, S.T., M.T.     & \\
      NIP: 19230323 197706 1 001        & \\
      &  \\                             & (Ko-Pembimbing)\\
      &  \\
      % Ubah kalimat-kalimat berikut dengan nama dan NIP dosen penguji pertama
      Dr. Galileo Galilei, S.T., M.Sc.  & \\
      NIP: 15640215 164201 1 001        & \\
      &  \\                             & (Penguji I)\\
      &  \\
      % Ubah kalimat-kalimat berikut dengan nama dan NIP dosen penguji kedua
      Friedrich Nietzsche, S.T., M.Sc.  & \\
      NIP: 18441015 190008 1 001        & \\
      &  \\                             & (Penguji II)\\ 
      &  \\
      % Ubah kalimat-kalimat berikut dengan nama dan NIP dosen penguji ketiga
      Alan Turing, ST., MT.             & \\
      NIP: 19120623 195406 1 001        & \\
      & \\                              & (Penguji III)\\
    \end{tabularx}
  \endgroup

  \vspace{4ex}

  \begin{center}
    % Ubah text dibawah menjadi tempat dan tanggal
    \textbf{SURABAYA} \\
    \textbf{Mei, 2077}
  \end{center}
\endgroup
