\begin{center}
	\large
  \textbf{\emph{ANTI-GRAVITY} BASED ENERGY CALCULATION ON OUTER SPACE ROCKETS}
\end{center}
% Menyembunyikan nomor halaman
\thispagestyle{empty}

\begin{flushleft}
    \setlength{\tabcolsep}{0pt}
    \bfseries
    \begin{tabular}{lc@{\hspace{6pt}}l}
    Student Name / NRP&: &Elon Reeve Musk / 0123204000001\\
    Department&: &Aerospace Engineering FTD - ITS\\
    Advisor&: &1. Nikola Tesla, S.T., M.T.\\
    & & 2. Wernher von Braun, S.T., M.T.\\
    \end{tabular}
    \vspace{4ex}
\end{flushleft}
\textbf{Abstract}

% Isi Abstrak
Suspension is an important component in vehicles because it plays an important role in
maintaining comfort and safety while driving. A new idea was introduced, namely, Series
Active Variable Geometry Suspension (SAVGS), where this suspension system has better
performance than passive suspension and can overcome the weaknesses of active suspension.
Research continues to improve the performance of SAVGS. The effect of linkage length (single
link) on SAVGS performance, especially comfort and stability, is studied. A quarter car is used
to model the dynamics of the vehicle suspension system. The effect of single link length is
analyzed in the form of stiffness coefficient and damping coefficient. The linear model is used
to design the state-feedback control system (LQR). The performance of the control system was
tested on a nonlinear model made using Simscape Multibody. The simulation results show that
the longer the single link used, the greater the vehicle's comfort and stability. However, the
longer the single link required more considerable control input.

\vspace{2ex}
\noindent
\textbf{Keywords: \emph{Rocket, Anti-gravity, Meong}}