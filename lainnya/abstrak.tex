\begin{center}
	\large
  \textbf{KALKULASI ENERGI PADA ROKET LUAR ANGKASA BERBASIS \emph{ANTI-GRAVITASI}}
\end{center}
\addcontentsline{toc}{chapter}{ABSTRAK}
% Menyembunyikan nomor halaman
\thispagestyle{empty}

\begin{flushleft}
    \setlength{\tabcolsep}{0pt}
    \bfseries
    \begin{tabular}{ll@{\hspace{6pt}}l}
    Nama Mahasiswa / NRP&:& Elon Reeve Musk / 0123204000001\\
    Departemen&:& Teknik Dirgantara FTD - ITS\\
    Dosen Pembimbing&:& 1. Nikola Tesla, S.T., M.T.\\
    & & 2. Wernher von Braun, S.T., M.T.\\
    \end{tabular}
    \vspace{4ex}
\end{flushleft}
\textbf{Abstrak}

% Isi Abstrak
Suspensi merupakan komponen penting pada kendaraan bermotor karena berperan
penting dalam menjaga kenyamanan dan keamanan saat berkendara. Sebuah ide baru
diperkenalkan yaitu, Series Active Variable Geometry Suspension (SAVGS), dimana sistem
suspensi ini memiliki performa yang lebih baik dari suspensi pasif dan dapat mengatasi
kelemahan dari suspensi aktif. Penelitian terus dilakukan guna meningkatkan performa dari
SAVGS. Pada penelitian ini akan dipelajari pengaruh panjang linkage (single link) terhadap
performa kendaraan khususnya kenyamanan dan stabilitas. Model seperempat kendaraan
digunakan untuk memodelkan dinamika sistem suspensi kendaraan. Pengaruh panjang single
link dianalisis dalam bentuk koefisien kekakuan dan koefisien peredam. Model linier digunakan
untuk merancang state-feedback control system (LQR). Kinerja sistem kendali diuji pada model
nonlinier yang dibuat dengan menggunakan Simscape Multibody. Hasil simulasi menunjukkan
bahwa semakin panjang single link yang digunakan maka kenyamanan dan stabilitas kendaraan
semakin besar. Namun, semakin panjang single link diperlukan input kontrol yang lebih besar.

\vspace{2ex}
\noindent
\textbf{Kata Kunci: \emph{Roket, Anti-gravitasi, Meong}}